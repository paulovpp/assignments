\documentclass[a4paper, 11pt]{article}
\usepackage{header}
\usepackage{amsmath,amssymb,bm}
\usepackage[brazilian]{babel}
\usepackage{enumerate}

% \usepackage{arial}
% \renewcommand{\familydefault}{\sfdefault}

% \usepackage{helvet}
% \renewcommand{\familydefault}{\sfdefault}

% Metadata
\date{\today}
\setmodule{Lógica proposicional}
\setterm{Semestre 1, 2022}

\title{Atividade 2}
\setmembername{Paulo Vinicius Pereira Pinheiro}  % Fill name
\setmemberuid{Lógica Matemática} 

\begin{document}
    \normalfont
    \header{}

    \begin{question}
        Avaliar as proposições para verificação de um número primo.
    \end{question}
    
    \begin{answer}[Resolução]
        \begin{enumerate}[1.]
            \item Avaliar o tamanho (em bits) do valor. Valores maiores que 25 bits são impraticáveis atualmente. Valores acima de 200 bits possuem tempo eterno de processamento. 
            \item Calcular a raiz quadrada do número primo.
            \item Avaliar se o mesmo é divisível por todos os primos anteriores ao valor de sua raiz quadrada.
            \item Criar vetor (array) com sequência de valores (excluindo o 1) primos anteriores para teste.
            \item Executar a sequência de testes. 
            \item Verificar se o valor $n$ é divisível pelos primos anteriores.
        \end{enumerate}
        Proposições: \\[4pt]
        $p=$ tamanho\_em\_bits(n) $>25$ menor que 
    \end{answer}

    \begin{question}
        Considere as fórmulas $H_n(p, q, r)$ a seguir e desenvolva suas respectivas tabelas verdade.
        \begin{enumerate}[\bf \quad a.]
            \item $H_1 = (\lnot p \lor q)~\leftrightarrow~(p \rightarrow q)$
            \item $H_2 = p \rightarrow ((q \rightarrow r) \rightarrow ((p \rightarrow R) \rightarrow (p \rightarrow R)))$ 
            \item $H_3 = (p \rightarrow \n q) \leftrightarrow \lnot p $
            \item $H_4 = (q \rar \n p) $
            \item $H_5 = (p \rar (q \rar r)) \lr ((p \e q) \rar r) $
            \item $H_6 = (false \rar q) \lr R$
            \item $H_7 = (p \lr \n q) \lr q \rar p$
            \item $H_8 = (p \lr \n q) \rar \n p \e q$
        \end{enumerate}
    \end{question}
    
\end{document}
