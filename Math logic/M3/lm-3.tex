\documentclass[a4paper, 11pt]{article}
\usepackage{header}
\usepackage{amsmath,amssymb,bm}
\usepackage[brazilian]{babel}
% \usepackage{enumerate}
\usepackage{enumitem}


% \usepackage{arial}
% \renewcommand{\familydefault}{\sfdefault}

% \usepackage{helvet}
% \renewcommand{\familydefault}{\sfdefault}

% Metadata
\date{\today}
\setmodule{Lógica proposicional para AVP1}
\setterm{Semestre 1, 2022}

\title{Atividade 3}
\setmembername{Paulo Vinicius Pereira Pinheiro}  % Fill name
\setmemberuid{Lógica Matemática} 

\begin{document}
    \normalfont
    \header{}

    \begin{question}
        Sejam as proposições p: está frio e q: está chovendo. Traduzir para a linguagem corrente as seguintes proposições:
        \vspace{-3mm}
        \begin{enumerate}[itemsep=-1mm]
            \item $\n p \ou q$
            \item $q \lr \n p$
            \item $\n p \e \n q$
            \item $(p \e \n q) \rar p$
        \end{enumerate}
    \end{question}

    \begin{question}
        Sejam as proposições p: Jorge é rico e q: Carlos é feliz. Traduzir para linguagem corrente as seguintes proposições:
        \vspace{-3mm}
        \begin{enumerate}[itemsep=-1mm]
            \item $p \rar q$
            \item $\n p \ou \n q$
            \item $\n \n p$
            \item $\n (\n p \e \n q)$
        \end{enumerate}
    \end{question}

    \begin{question}
        Sejam as proposições p: Marcos é alto e q: Marcos é elegante. Traduzir para a linguagem simbólica as seguintes proposições:
        \vspace{-3mm}
        \begin{enumerate}[itemsep=-1mm]
            \item Marcos é alto e elegante.
            \item Marcos é alto, mas não é elegante.
            \item Não é verdade que Marcos é baixo e elegante.
            \item Marcos é alto ou é baixo e elegante.
        \end{enumerate}
    \end{question}
    
    \begin{question}
        Dadas as seguintes proposições:
        \vspace{-3mm}
        \begin{enumerate}[itemsep=-1mm]
            \item[] p: o número 596 é divisível por 2.
            \item[] q: o número 596 é divisível por 4.
            \item[] r: o número 596 é divisível por 3.
        \end{enumerate}
        Traduzir para a linguagem simbólica:
        \vspace{-3mm}
        \begin{enumerate}[itemsep=-1mm, label = {\alph*.}]
            \item É falso que número 596 é divisível por 2 e por 3, ou o número 596 não é divisível por 4.
            \item O número 596 não é divisível por 2 ou por 4, mas é divisível por 3.
            \item Se não é verdade que o número 596 é divisível por 3, então ele é divisível por 2 e não por 4.
            \item É falso que o número 596 não é divisível por 2 e por 4, mas é divisível por 3 e por 2.
        \end{enumerate}
    \end{question}

    \begin{question}
        Sabendo-se que $V(p) = V(q) = V$ e $V(r) = V(s) = F$, determine os valores lógicos das seguintes proposições:
        \vspace{-3mm}
        \begin{enumerate}[itemsep=-1mm]
            \item $(p \rar (q \e \n r)) \ou (p \rar (r \rar q))$
            \item $(q \ou r) \lr (\n q \rar r)$
            \item $(\n p \e \n(r \e \n s))$
            \item $\n (q \rar (\n p \lr s))$
        \end{enumerate}
    \end{question}

    \begin{question}
        Demonstrar utilizando o método dedutivo a equivalência abaixo:
        \vspace*{-3mm}
        \begin{equation*}
            p \lr(q \e p) \Longleftrightarrow  p \rar q
        \end{equation*}
    \end{question}

    \begin{question}
        Defina se as relações abaixo são tautológicas, contraditórias ou contingentes.
        \vspace*{-3mm}
        \begin{enumerate}[itemsep=-1mm]
            \item $(p \rar (p \rar \n q)) \e q$
            \item $(p \rar q) \rar (p \ou r \rar q \ou r)$
        \end{enumerate}
    \end{question}



    \begin{question}
        Considere as fórmulas $H_n(p, q, r)$ a seguir e desenvolva suas respectivas tabelas verdade.
        \vspace{-3mm}
        % \begin{enumerate}[\bf \quad a.]
        \begin{enumerate}[itemsep=-1mm]
            \item[] $H_1 = (\lnot p \lor q)~\leftrightarrow~(p \rightarrow q)$
            \item[] $H_2 = p \rightarrow ((q \rightarrow r) \rightarrow ((p \rightarrow r) \rightarrow (p \rightarrow r)))$ 
            \item[] $H_3 = (p \rightarrow \n q) \leftrightarrow \lnot p $
            \item[] $H_4 = (q \rar \n p) $
            \item[] $H_5 = (p \rar (q \rar r)) \lr ((p \e q) \rar r) $
            \item[] $H_6 = (false \rar q) \lr r$
            \item[] $H_7 = (p \lr \n q) \lr q \rar p$
            \item[] $H_8 = (p \lr \n q) \rar \n p \e q$
        \end{enumerate}
    \end{question}
    
\end{document}
