\documentclass[a4paper, 11pt]{article}
\usepackage{header}
\usepackage{amsmath,amssymb,bm}
\usepackage[brazilian]{babel}
% \usepackage{enumerate}
\usepackage{enumitem}


% \usepackage{arial}
% \renewcommand{\familydefault}{\sfdefault}

% \usepackage{helvet}
% \renewcommand{\familydefault}{\sfdefault}

% Metadata
\date{\today}
\setmodule{Equações Lógicas para o TDE 4}
\setterm{Semestre 1, 2022}

\title{Atividade 4}
\setmembername{Paulo Vinicius Pereira Pinheiro}  % Fill name
\setmemberuid{Lógica Matemática} 

\begin{document}
    \normalfont
    \header{}

    \begin{question}
        Considere as equações lógicas abaixo e utilize um simulador computacional para simular a equação correspondente ao último dígito de sua matrícula. Em seguida, desenvolva também sua tabela verdade correspondente.
        \vspace{-3mm}
        % \begin{enumerate}[\bf \quad a.]
        \begin{enumerate}
            \setcounter{enumi}{-1}
            \item $S_0 = (B \oplus (C+\overline{A+C}))+(A + (\overline{B.C.\overline{D}}))$
            \item $S_1 = [(\overline{\overline{A}+B}) + (\overline{\overline{C}.D})].\overline{D} + (\overline{B+C})$
            \item $S_2 = \overline{[\overline{A}.B.(\overline{B.C}).(\overline{B+D})]}\oplus (A+D)$
            \item $S_3 = \overline{[\overline{(\overline{A}.B)} + (\overline{A + \overline{B}})+\overline{C}]}.(C+D)$
            \item $S_4 = [\overline{(\overline{(A.C)}+D+B)}]+C.(\overline{A.C.D})$
            \item $S_5 = [\overline{\overline{(B.D+A)}(C.D + \overline{(A+B)})}]$
            \item $S_6 = (A.B)+[\overline{\overline{(A.\overline{C}.B)}+(B+C+\overline{D})}]$
            \item $S_7 = [\overline{(\overline{A}+B+\overline{C})}.D]+[C.(\overline{\overline{A}+B+\overline{C}})]$
            \item $S_8 = [\overline{\overline{A} + \overline{B}.\overline{(C+\overline{D})}+A.\overline{B}.\overline{C}}]\oplus (B.D)$
            \item $S_9 = A + \overline{B.D} + \overline{C}.[\overline{A.B+\overline{C.D} \oplus \overline{(B+C)}}]$
        \end{enumerate}
    \end{question}
    
\end{document}
