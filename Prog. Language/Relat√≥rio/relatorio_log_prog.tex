\documentclass[
	% -- opções da classe memoir --
	12pt,				% tamanho da fonte
	openright,			% capítulos começam em pág ímpar (insere página vazia caso preciso)
	oneside,			% para impressão em verso e anverso. Oposto a oneside
	a4paper,			% tamanho do papel. 
	% -- opções da classe abntex2 --
	chapter=TITLE,		% títulos de capítulos convertidos em letras maiúsculas
	section=TITLE,		% títulos de seções convertidos em letras maiúsculas
	%subsection=TITLE,	% títulos de subseções convertidos em letras maiúsculas
	%subsubsection=TITLE,% títulos de subsubseções convertidos em letras maiúsculas
	% -- opções do pacote babel --
	english,			% idioma adicional para hifenização
	french,				% idioma adicional para hifenização
	spanish,			% idioma adicional para hifenização
	brazil				% o último idioma é o principal do documento
	]{abntex2}

% ---
% Pacotes básicos 
% ---
\usepackage{lmodern}			% Usa a fonte Latin Modern			
\usepackage[T1]{fontenc}		% Selecao de codigos de fonte.
\usepackage[utf8]{inputenc}		% Codificacao do documento (conversão automática dos acentos)
\usepackage{lastpage}			% Usado pela Ficha catalográfica
\usepackage{indentfirst}		% Indenta o primeiro parágrafo de cada seção.
\usepackage{color}				% Controle das cores
\usepackage{graphicx}			% Inclusão de gráficos
\usepackage{microtype} 			% para melhorias de justificação
\usepackage[brazilian,hyperpageref]{backref}	 % Paginas com as citações na bibl
\usepackage[alf]{abntex2cite}	% Citações padrão Absent
\usepackage{lipsum}
\usepackage{listings}
\usepackage{xcolor}
\usepackage{fancyhdr}
% \usepackage{subfigure}

\definecolor{codegreen}{rgb}{0,0.6,0}
\definecolor{codegray}{rgb}{0.5,0.5,0.5}
\definecolor{codepurple}{rgb}{0.58,0,0.82}
\definecolor{backcolour}{rgb}{0.95,0.95,0.92}
\definecolor{blue}{RGB}{41,5,195}

\lstdefinestyle{mystyle}{
    backgroundcolor=\color{backcolour},   
    commentstyle=\color{codegreen},
    keywordstyle=\color{blue},
    numberstyle=\tiny\color{codegray},
    stringstyle=\color{red},
    basicstyle=\ttfamily\footnotesize,
    breakatwhitespace=false,         
    breaklines=true,                 
    captionpos=b,                    
    keepspaces=true,                 
    numbers=left,                    
    numbersep=0.5pt,                  
    showspaces=false,                
    showstringspaces=false,
    showtabs=false,                  
    tabsize=2
}
\lstset{style=mystyle}

% alterando o aspecto da cor azul


% informações do PDF
\makeatletter
\hypersetup{
    %pagebackref=true,
	pdftitle={\@title}, 
	pdfauthor={{Name}},
	pdfsubject={\imprimirpreambulo},
	pdfcreator={LaTeX with abnTeX2},
	pdfkeywords={abnt}{latex}{abntex}{abntex2}{trabalho acadêmico}, 
	colorlinks=true,       		% false: boxed links; true: colored links
	linkcolor=blue,          	% color of internal links
	citecolor=blue,        		% color of links to bibliography
	filecolor=magenta,      		% color of file links
	urlcolor=blue,
	bookmarksdepth=4
}
\makeatother

% O tamanho do parágrafo é dado por:
\setlength{\parindent}{1.3cm}

% Controle do espaçamento entre um parágrafo e outro:
\setlength{\parskip}{0.2cm}  % tente também \onelineskip

% compila o índice
% ---
\makeindex
% ----
% Início do documento
% ----
\begin{document}

% Retira espaço extra obsoleto entre as frases.
\frenchspacing 

% Informações de dados para CAPA
% Capa sem o uso de \capa
\begin{capa}
	\begin{center}
		\ABNTEXchapterfont\large{
			\includegraphics[scale=1]{Imgs/logo.png}\\
			CENTRO UNIVERSITÁRIO INTERNACIONAL UNINTER\\
			ESCOLA SUPERIOR POLITÉCNICA\\
			BACHARELADO EM ENGENHARIA DA COMPUTAÇÃO\\
			DISCIPLINA DE LINGUAGEM DE PROGRAMAÇÃO
		}
		\vfill

		\ABNTEXchapterfont\bfseries\LARGE{
			Atividade Prática
		}
	
		\vspace*{2cm}

		\begin{flushright}
			\ABNTEXchapterfont\bfseries\normalsize{
				Autor: Paulo Vinicius Pereira Pinheiro\\
				No. RU: 3760288\\
				Prof. Sandro de Araújo
			}
		\end{flushright}
		
		\vfill

		% \ABNTEXchapterfont\large{
		\ABNTEXchapterfont\normalsize{
			Juazeiro do Norte - Ceará\\
			2022			
		}
	\end{center}
\end{capa}


\pdfbookmark[0]{\listfigurename}{lof}
\listoffigures*
\cleardoublepage

\lstlistoflistings
\pagebreak


% inserir o sumario
% ---
\pdfbookmark[0]{\contentsname}{toc}
\tableofcontents*
% \cleardoublepage
% ---



% ----------------------------------------------------------
% ELEMENTOS TEXTUAIS
% ----------------------------------------------------------
\textual
% \setcounter{page}{}
% ----------------------------------------------------------
% Introdução (exemplo de capítulo sem numeração, mas presente no Sumário)
% ----------------------------------------------------------
\chapter*[INTRODUÇÃO]{\bfseries{Introdução}}
\addcontentsline{toc}{chapter}{INTRODUÇÃO}
% ----------------------------------------------------------

Este documento e seu código-fonte são exemplos de referência de uso da classe
\textsf{abntex2} e do pacote \textsf{abntex2cite}. O documento 
exemplifica a elaboração de trabalho acadêmico (tese, dissertação e outros do
gênero) produzido conforme a ABNT NBR 14724:2011 \emph{Informação e documentação
- Trabalhos acadêmicos - Apresentação}.

A expressão ``Modelo Canônico'' é utilizada para indicar que \abnTeX\ não é
modelo específico de nenhuma universidade ou instituição, mas que implementa tão
somente os requisitos das normas da ABNT. Uma lista completa das normas

Sinta-se convidado a participar do projeto \abnTeX! Acesse o site do projeto em
\url{http://abntex2.googlecode.com/}. Também fique livre para conhecer,
estudar, alterar e redistribuir o trabalho do \abnTeX, desde que os arquivos
modificados tenham seus nomes alterados e que os créditos sejam dados aos
autores originais, nos termos da ``The \LaTeX\ Project Public
License''\footnote{\url{http://www.latex-project.org/lppl.txt}}.

Encorajamos que sejam realizadas customizações específicas deste exemplo para
universidades e outras instituições --- como capas, folha de aprovação, etc.
Porém, recomendamos que ao invés de se alterar diretamente os arquivos do
\abnTeX, distribua-se arquivos com as respectivas customizações.
Isso permite que futuras versões do \abnTeX~não se tornem automaticamente
incompatíveis com as customizações promovidas. Consulte


Esperamos, sinceramente, que o \abnTeX\ aprimore a qualidade do trabalho que
você produzirá, de modo que o principal esforço seja concentrado no principal:
na contribuição científica.


\chapter{\bfseries{Prática}}


\section{Enunciado}

	Escreva um algoritmo em linguagem C que atenda aos seguintes requisitos:

	\begin{itemize}[itemsep = -1mm]
		\item Os campos de um registro devem armazenar o Nome, dia e mês de aniversário.
		\item Solicite ao usuário que digite 08 registros.
		\item Os registros devem ser armazenados em um vetor.
		\item Através do ponteiro para o vetor de registro mostre em cada um dos meses do ano quem são as pessoas que fazem aniversário.
	\end{itemize}

	Para demonstrar o funcionamento faça as capturas de tela do terminal utilizando seu nome completo e o seu dia e mês de aniversário em um dos registros de entrada solicitado.

\section{Código-Fonte da Solução}

	Para o desenvolvimento da solução foi utilizada a versão 2022 do Microsoft Visual Studio. Abaixo, está o código-fonte da solução.

	\lstinputlisting[language=C++, caption = Código fonte da prática 1 - registro de pessoas e aniversários.]{source.cpp}

	O código apresentado está dividido em duas partes: a primeira realiza a aquisição dos registros e suas datas de aniversário. Nesta parte, através da instrução de sistema \emph{system("cls")}, o sistema limpa a tela após cada inserção de dados. Algumas telas são apresentadas na seção de discussão dos resultados.
	
	A segunda parte, responsável pela seleção e exibição dos registros, utiliza um vetor de ponteiros na implementação. 


\section{Discussão dos resultados}

	Inicia-se a apresentação dos resultados apresentando a tela inicial do programa no terminal, como mostra a Figura \ref{fig:ex1.1}.

	\begin{figure}[htpb]
		\centering
		\caption{Tela inicial do programa.}
		\includegraphics[scale=0.75]{Imgs/ex1.1.PNG}
		\label{fig:ex1.1}
	\end{figure}

	O título incluído no programa é apresentado em todas as telas. Conforme elencado anteriormente, a aquisição dos registros é realizado em tela individual. Acima da solicitação do nome, percebe-se que o usuário pode identificar qual registro está sendo inserido. Nas Figuras \ref{fig:ex1.23} e \ref{fig:ex1.45} são apresentadas algumas das telas de inserção de dados.

	\begin{figure}[htb]
		\begin{center}
			\caption{Telas com os campos de inserção de dados.}
			\label{fig:ex1.23}
			\includegraphics[height=3.5cm]{Imgs/ex1.2.PNG} \quad
			\includegraphics[height=3.5cm]{Imgs/ex1.3.PNG} \quad
		\end{center}
	\end{figure}

	\vspace*{-8mm}

	\begin{figure}[htb]
		\begin{center}
			\caption{Mais telas com campos de inserção de dados.}
			\label{fig:ex1.45}
			\includegraphics[height=3.5cm]{Imgs/ex1.4.PNG} \quad
			\includegraphics[height=3.5cm]{Imgs/ex1.5.PNG} \quad
		\end{center}
	\end{figure}

	\vspace*{-4mm}

	Na figura \ref{fig:ex1.6} está a tela final com a captura total dos dados e sua separação pelos meses de aniversário. Além do solicitado, foi incluído também no resultado final o dia em que o funcionário tem seu aniversário.

	\begin{figure}[ht]
		\begin{center}
			\caption{Tela final com a segregação dos dados por mês.} 
			\includegraphics[height=9cm]{Imgs/ex1.6.PNG}
			\label{fig:ex1.6}
		\end{center}
	\end{figure}


\chapter{Lectus lobortis condimentum}
% ---

% ---
\section{Vestibulum ante ipsum primis in faucibus orci luctus et ultrices
posuere cubilia Curae}
% ---

\lipsum[21-22]

% ---
% segundo capitulo de Resultados
% ---
\chapter{Nam sed tellus sit amet lectus urna ullamcorper tristique interdum
elementum}
% ---

% ---
\section{Pellentesque sit amet pede ac sem eleifend consectetuer}
% ---

\lipsum[24]

% ----------------------------------------------------------
% Finaliza a parte no bookmark do PDF
% para que se inicie o bookmark na raiz
% e adiciona espaço de parte no Sumário
% ----------------------------------------------------------
%\phantompart

% ---
% Conclusão (outro exemplo de capítulo sem numeração e presente no sumário)
% ---
\chapter*[Conclusão]{Conclusão}
\addcontentsline{toc}{chapter}{Conclusão}
% ---

\lipsum[31-33]

% \bibliography{abntex2-modelo-references}

\end{document}
