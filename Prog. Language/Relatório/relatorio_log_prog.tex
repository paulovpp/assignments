\documentclass[
	% -- opções da classe memoir --
	12pt,				% tamanho da fonte
	openright,			% capítulos começam em pág ímpar (insere página vazia caso preciso)
	oneside,			% para impressão em verso e anverso. Oposto a oneside
	a4paper,			% tamanho do papel. 
	% -- opções da classe abntex2 --
	chapter=TITLE,		% títulos de capítulos convertidos em letras maiúsculas
	section=TITLE,		% títulos de seções convertidos em letras maiúsculas
	%subsection=TITLE,	% títulos de subseções convertidos em letras maiúsculas
	%subsubsection=TITLE,% títulos de subsubseções convertidos em letras maiúsculas
	% -- opções do pacote babel --
	english,			% idioma adicional para hifenização
	french,				% idioma adicional para hifenização
	spanish,			% idioma adicional para hifenização
	brazil				% o último idioma é o principal do documento
	]{abntex2}

% ---
% Pacotes básicos 
% ---
\usepackage{lmodern}			% Usa a fonte Latin Modern			
\usepackage[T1]{fontenc}		% Selecao de codigos de fonte.
\usepackage[utf8]{inputenc}		% Codificacao do documento (conversão automática dos acentos)
\usepackage{lastpage}			% Usado pela Ficha catalográfica
\usepackage{indentfirst}		% Indenta o primeiro parágrafo de cada seção.
\usepackage{color}				% Controle das cores
\usepackage{graphicx}			% Inclusão de gráficos
\usepackage{microtype} 			% para melhorias de justificação
\usepackage[brazilian,hyperpageref]{backref}	 % Paginas com as citações na bibl
\usepackage[alf]{abntex2cite}	% Citações padrão Absent
\usepackage{lipsum}
\usepackage{listingsutf8}		% Listings com codigos em utf8
\usepackage{xcolor}
\usepackage{fancyhdr}
\usepackage{setspace}
\usepackage{inconsolata} % Swedish encoding in lstlisting
\usepackage{siunitx}

\definecolor{codegreen}{rgb}{0,0.6,0}
\definecolor{codegray}{rgb}{0.5,0.5,0.5}
\definecolor{codepurple}{rgb}{0.58,0,0.82}
\definecolor{backcolour}{rgb}{0.95,0.95,0.92}
% \definecolor{blue}{RGB}{41,5,195}

\lstdefinestyle{mystyle}{
    backgroundcolor=\color{backcolour},   
    commentstyle=\color{codegreen},
    keywordstyle=\color{blue},
	lineskip={-1pt},
	basicstyle={\ttfamily\singlespacing},
	% basicstyle=\linespread{0.5},
    numberstyle=\tiny\color{codegray},
    stringstyle=\color{red!75!green},
    basicstyle=\ttfamily\footnotesize,
    breakatwhitespace=false,         
    breaklines=true,                 
    captionpos=b,                    
    keepspaces=true,                 
    numbers=left,                    
    numbersep=2pt,                  
    showspaces=false,                
    showstringspaces=false,
    showtabs=false,                  
    tabsize=2
}
\lstset{style=mystyle}

\lstset{
	inputencoding=utf8,
    extendedchars=true,
    literate={á}{{\'a}}1 {à}{{\`a}}1 {ã}{{\~a}}1 {é}{{\'e}}1 {ê}{{\^e}}1 {ë}{{\"e}}1 {í}{{\'i}}1 {ç}{{\c{c}}}1 {Ç}{{\c{C}}}1 {õ}{{\~o}}1 {ó}{{\'o}}1 {ô}{{\^o}}1 {ú}{{\'u}}1 {Ã}{{\~A}}1 {â}{{\^a}}1
}

% informações do PDF
\makeatletter
\hypersetup{
    %pagebackref=true,
	pdftitle={\@title}, 
	pdfauthor={{Name}},
	pdfsubject={\imprimirpreambulo},
	pdfcreator={LaTeX with abnTeX2},
	pdfkeywords={abnt}{latex}{abntex}{abntex2}{trabalho acadêmico}, 
	colorlinks=true,       		% false: boxed links; true: colored links
	linkcolor=blue,          	% color of internal links
	citecolor=blue,        		% color of links to bibliography
	filecolor=magenta,      		% color of file links
	urlcolor=blue,
	bookmarksdepth=4
}
\makeatother

% O tamanho do parágrafo é dado por:
\setlength{\parindent}{1.3cm}

% Controle do espaçamento entre um parágrafo e outro:
\setlength{\parskip}{0.2cm}  % tente também \onelineskip

% compila o índice
% ---
\makeindex
% ----
% Início do documento
% ----
\begin{document}

% Retira espaço extra obsoleto entre as frases.
\frenchspacing 

% Informações de dados para CAPA
% Capa sem o uso de \capa
\begin{capa}
	\begin{center}
		\ABNTEXchapterfont\large{
			\includegraphics[scale=1]{Imgs/logo.png}\\
			CENTRO UNIVERSITÁRIO INTERNACIONAL UNINTER\\
			ESCOLA SUPERIOR POLITÉCNICA\\
			BACHARELADO EM ENGENHARIA DA COMPUTAÇÃO\\
			DISCIPLINA DE LINGUAGEM DE PROGRAMAÇÃO
		}
		\vfill

		\ABNTEXchapterfont\bfseries\LARGE{
			Atividade Prática
		}
	
		\vspace*{2cm}

		\begin{flushright}
			\ABNTEXchapterfont\bfseries\normalsize{
				Autor: Paulo Vinicius Pereira Pinheiro\\
				No. RU: 3760288\\
				Prof. Sandro de Araújo\\
				Prof. Winston Sen Lun Fung
			}
		\end{flushright}
		
		\vfill

		% \ABNTEXchapterfont\large{
		\ABNTEXchapterfont\normalsize{
			Juazeiro do Norte - Ceará\\
			2022			
		}
	\end{center}
\end{capa}


\pdfbookmark[0]{\listfigurename}{lof}
\listoffigures*
\cleardoublepage

\lstlistoflistings
\pagebreak


% inserir o sumario
% ---
\pdfbookmark[0]{\contentsname}{toc}
\tableofcontents*
% \cleardoublepage
% ---



% ----------------------------------------------------------
% ELEMENTOS TEXTUAIS
% ----------------------------------------------------------
\textual
% \setcounter{page}{}
% ----------------------------------------------------------
% Introdução (exemplo de capítulo sem numeração, mas presente no Sumário)
% ----------------------------------------------------------
\chapter*[INTRODUÇÃO]{\bfseries{Introdução}}
\addcontentsline{toc}{chapter}{INTRODUÇÃO}
% ----------------------------------------------------------

A linguagem de programação C foi desenvolvida por Dennis Ritchie na década de 70 como um aprimoramento da linguagem B. No entanto, somente em 1983, o ANSI (\emph{AMERICAN NATIONAL STANDARD INSTITUTE}) publicou a primeira versão da linguagem C padronizada para sanar as discrepâncias encontradas na linguagem.

Com a popularidade dos microcomputadores, logo a linguagem se tornou universalmente utilizada. Inúmeros foram os compiladores criados para a linguagem C, que foram usados em diversos sistemas operacionais.

C frequentemente é chamada de linguagem de médio nível, pois é uma linguagem de baixo nível, com menos recursos e mais pouco espaço de memória. Isso não significa que ela seja menos poderosa que as demais, como PASCAL e BASIC. Tampouco se deve comparar a mesma com ASSEMBLY. Ela é conhecida da forma mencionada acima pois engloba recursos de linguagens de baixo nível e de alto nível.

Este trabalho é fruto do desenvolvimento da atividade prática solicitada na disciplina de Linguagem de programação. O mesmo está divido em cinco partes, cada uma correspondendo a uma prática solicitada. Tal formato difere do tradicional modelo de relatório pois tenta simplificar a forma de apresentação dos resultados. E não custa enfatizar que o mesmo segue o modelo solicitado pelo professor.

\chapter[\bfseries{PRÁTICA 1}]{\bfseries{Prática}}

\section[ENUNCIADO]{Enunciado}

	Escreva um algoritmo em linguagem C que atenda aos seguintes requisitos:

	\begin{itemize}[itemsep = -1mm]
		\item Os campos de um registro devem armazenar o Nome, dia e mês de aniversário.
		\item Solicite ao usuário que digite 08 registros.
		\item Os registros devem ser armazenados em um vetor.
		\item Através do ponteiro para o vetor de registro mostre em cada um dos meses do ano quem são as pessoas que fazem aniversário.
	\end{itemize}

	Para demonstrar o funcionamento faça as capturas de tela do terminal utilizando seu nome completo e o seu dia e mês de aniversário em um dos registros de entrada solicitado.

\section[CÓDIGO-FONTE DA SOLUÇÃO]{Código-Fonte da Solução}

	Para o desenvolvimento da solução foi utilizada a versão 2022 do Microsoft Visual Studio. Abaixo, está o código-fonte da solução.

	\lstinputlisting[language=C++, caption = Código fonte da prática 1 - registro de pessoas e aniversários., captionpos=t]{Ex1.cpp}	
	
	O código apresentado está dividido em duas partes: a primeira realiza a aquisição dos registros e suas datas de aniversário. Nesta parte, através da instrução de sistema \emph{system("cls")}, o sistema limpa a tela após cada inserção de dados. Algumas telas são apresentadas na seção de discussão dos resultados.
	
	A segunda parte, responsável pela seleção e exibição dos registros, utiliza um vetor de ponteiros na implementação. 


\section[DISCUSSÃO DOS RESULTADOS]{Discussão dos resultados}

	Inicia-se a apresentação dos resultados apresentando a tela inicial do programa no terminal, como mostra a Figura \ref{fig:ex1.1}.

	\begin{figure}[htpb]
		\centering
		\caption{Tela inicial do programa.}
		\includegraphics[scale=0.75]{Imgs/ex1.1.PNG}
		\label{fig:ex1.1}
	\end{figure}

	O título incluído no programa é apresentado em todas as telas. Conforme elencado anteriormente, a aquisição dos registros é realizado em tela individual. Acima da solicitação do nome, percebe-se que o usuário pode identificar qual registro está sendo inserido. Nas Figuras \ref{fig:ex1.23} e \ref{fig:ex1.45} são apresentadas algumas das telas de inserção de dados.

	\begin{figure}[htb]
		\begin{center}
			\caption{Telas com os campos de inserção de dados.}
			\label{fig:ex1.23}
			\includegraphics[height=3.5cm]{Imgs/ex1.2.PNG} \quad
			\includegraphics[height=3.5cm]{Imgs/ex1.3.PNG} \quad
		\end{center}
	\end{figure}

	\vspace*{-8mm}

	\begin{figure}[htb]
		\begin{center}
			\caption{Outras telas complementares com os campos de inserção de dados.}
			\label{fig:ex1.45}
			\includegraphics[height=3.5cm]{Imgs/ex1.4.PNG} \quad
			\includegraphics[height=3.5cm]{Imgs/ex1.5.PNG} \quad
		\end{center}
	\end{figure}

	% \vspace*{-4mm}

	Na figura~\ref{fig:ex1.6} está a tela final com a captura total dos dados e sua separação pelos meses de aniversário. Além do solicitado, foi incluído também no resultado final o dia em que o funcionário tem seu aniversário.

	\begin{figure}[ht]
		\begin{center}
			\caption{Tela final com a segregação dos dados por mês.} 
			\includegraphics[height=15cm]{Imgs/ex1.6.PNG}
			\label{fig:ex1.6}
		\end{center}
	\end{figure}


\chapter[\bfseries{PRÁTICA 2}]{\bfseries{Prática}}

\section[ENUNCIADO]{Enunciado}
	
	Faça um programa onde o usuário digita 3 informações a respeito de uma pessoa: Nome, endereço e telefone. Concatene essas três informações em uma única string e faça uma contagem de quantas letras do alfabeto estão presentes nesta string (considerando as redundâncias) e de dígitos numéricos. Os espaços e os caracteres de pontuação devem ser ignorados (as funções de contagem já fazem isso).

\section[CÓDIGO-FONTE DA SOLUÇÃO]{Código-Fonte da Solução}

	O código fonte desenvolvido como solução para o problema do enunciado está abaixo:

	\lstinputlisting[language=C++, caption = Código fonte da prática 2 - verificação e contagem de caracteres., captionpos=t]{Ex2.cpp}	
	
	O código fonte apresenta todos os itens solicitados. Dentro do laço for da linha 56, uma verificação é realizada por toda \emph{string} já concatenada (linhas 52-54). As funções \emph{isalpha} e \emph{isdigit} são utilizadas para verificar se o caractere está presente no alfabeto e no dígito, correspondentemente nas linhas 63 e 70. Além do solicitado, também foi desenvolvida a rotina para apresentação da soma dos caracteres alfanuméricos.

\section[DISCUSSÃO DOS RESULTADOS]{Discussão dos resultados}

	Na figura~\ref{fig:ex2.1}, verifica-se a tela de execução do programa com as informações de input e resultados já mostrados.

	\begin{figure}[ht]
		\begin{center}
			\caption{Tela final com a segregação dos dados por mês.} 
			\includegraphics[width=0.9\textwidth]{Imgs/ex2.1.PNG}
			\label{fig:ex2.1}
		\end{center}
	\end{figure}

	Conforme apresentado no código fonte e solicitado no enunciado, os caracteres numéricos e alfabéticos são contados, da mesma forma como todos os caracteres alfanuméricos e símbolos válidos.
% ---
\chapter[\bfseries{PRÁTICA 3}]{\bfseries{Prática}}

\section[ENUNCIADO]{Enunciado}

	Faça um programa C para calcular o número de lâmpadas 60 watts necessárias para um determinado cômodo. O programa deverá ler um conjunto de informações, tais como: tipo, largura e comprimento do cômodo. O programa termina quando o tipo de cômodo for igual -1. A tabela abaixo mostra, para cada tipo de cômodo, a quantidade de watts por metro quadrado.

	\begin{table}[htpb]
		\caption{Dados do problema.}
		\label{tab:table_2}
		\centering
		\begin{tabular}{c|c}
		\hline
		\multicolumn{1}{l|}{\textbf{TIPO CÔMODO}} & \multicolumn{1}{l}{\textbf{POTÊNCIA(\unit{W/m^2})}} \\ \hline
		0                                         & 12                                                                                    \\ \hline
		1                                         & 15                                                                                    \\ \hline
		2                                         & 18                                                                                    \\ \hline
		3                                         & 20                                                                                    \\ \hline
		4                                         & 22                                                                                    \\ \hline
		\end{tabular}
	\end{table}

\section[CÓDIGO-FONTE DA SOLUÇÃO]{Código-Fonte da Solução}

	O código fonte desenvolvido como solução para o problema do enunciado está abaixo:

	\lstinputlisting[language=C++, caption = Código fonte da prática 3 - adequação do número de lâmpadas a comodos., captionpos=t]{Ex3.cpp}	
	
	O código fonte apresenta todos os itens solicitados. Dentro do laço \emph{while} da linha 48, todos os inputs são solicitados e o laço se repete até o usuário digitar -1. A área do cômodo e a quantidade de lâmpadas são calculados em funções separadas da main, a partir da linha 72.

\section[DISCUSSÃO DOS RESULTADOS]{Discussão dos resultados}
	
	Na figura~\ref{fig:ex3.1}, verifica-se a tela de execução do programa com as informações de input e resultados já mostrados.
	
	\begin{figure}[htp]
		\begin{center}
			\caption{Tela inicial com a solicitação do tipo do cômodo.} 
			\includegraphics[width=0.9\textwidth]{Imgs/ex3.1.PNG}
			\label{fig:ex3.1}
		\end{center}
	\end{figure}
	
	A tela inicial, figura~\ref{fig:ex3.1}, conta com um título e apresenta as informações de input solicitadas. A figura XX apresenta a tela com o input dos dados solicitados e o resultado: a quantidade de lâmpadas necessárias para um espaço daquele tipo e com aquelas dimensões.

	\begin{figure}[htp]
		\begin{center}
			\caption{Tela final com a apresentação dos resultados.} 
			\includegraphics[width=0.8\textwidth]{Imgs/ex3.2.PNG}
			\label{fig:ex3.2}
		\end{center}
	\end{figure}

	Conforme apresentado no código fonte e solicitado no enunciado, o programa calcula a área do cômodo e a quantidade de lâmpadas necessárias para o cômodo escolhido e deve permanecer solicitando esses inputs continuamente até o usuário digitar -1. Na figura~\ref{fig:ex3.3} está apresentada a tela confirmando a condição de saída do programa.

	\begin{figure}[htp]
		\begin{center}
			\caption{Tela final com a apresentação dos resultados.} 
			\includegraphics[width=0.8\textwidth]{Imgs/ex3.3.PNG}
			\label{fig:ex3.3}
		\end{center}
	\end{figure}

\chapter[\bfseries{PRÁTICA 4}]{\bfseries{Prática}}

\section[ENUNCIADO]{Enunciado}
	
	Escreva em linguagem C um algoritmo que:

	\begin{itemize}
		\item Solicite ao usuário que digite o seu RU;
		\item Armazena cada dígito do Ru em uma posição de um vetor;
		\begin{table}[htpb]
			% \caption{Dados do problema.}
			% \label{tab:table_3}
			\centering
			\begin{tabular}{c|c|c|c|c|c|c|c|}
			\cline{2-8}
			\textbf{Vetor RU}                                                       & \textbf{\begin{tabular}[c]{@{}c@{}}Primeiro \\ dígito\end{tabular}} &   &   &   &   &   & \textbf{\begin{tabular}[c]{@{}c@{}}Último\\ dígito\end{tabular}} \\ \cline{2-8} 
			\textbf{\begin{tabular}[c]{@{}c@{}}Posição do \\ vetor RU\end{tabular}} & 0                                                                   & 1 & 2 & 3 & 4 & 5 & 6                                                                \\ \cline{2-8} 
			\end{tabular}
			\end{table}
		\item Utilizando uma função recursiva mostre o qual o valor do menor dígito inserido no vetor.
		\item Utilizando outra função recursiva mostre o qual o valor do maior dígito inserido no vetor
	\end{itemize}

\section[CÓDIGO-FONTE DA SOLUÇÃO]{Código-Fonte da Solução}

	O código fonte desenvolvido como solução para o problema do enunciado está abaixo:

	\lstinputlisting[language=C++, caption = Código fonte da prática 4 - verificação do menor e maior dígitos., captionpos=t]{Ex4.cpp}	
	
	O código fonte apresenta todos os itens solicitados. Após a função principal, estão inseridas as funções \emph{menor} e \emph{maior} para verificar o menor e o maior dígito do vetor RU.

\section[DISCUSSÃO DOS RESULTADOS]{Discussão dos resultados}
	
	Na figura~\ref{fig:ex4.1}, verifica-se a tela de execução do programa com as informações de input e resultados já mostrados.

	\begin{figure}[htp]
		\begin{center}
			\caption{Tela com a solicitação do vetor de entrada e resultados.} 
			\includegraphics[width=0.85\textwidth]{Imgs/ex4.1.PNG}
			\label{fig:ex4.1}
		\end{center}
	\end{figure}

\chapter[\bfseries{PRÁTICA 5}]{\bfseries{Prática}}

\section[ENUNCIADO]{Enunciado}
		
	Crie um programa, em linguagem C, que receba 7 registros contendo, Nome do Produto, Código do produto (numérico), valor do produto. Solicite que sejam digitados todos os dados de todos os registros e ao final salve-os em um arquivo.csv, utilize o ;(ponto e vírgula) para separador de campo. O nome do arquivo deve ser o seu número de RU.

\section[CÓDIGO-FONTE DA SOLUÇÃO]{Código-Fonte da Solução}

	O código fonte desenvolvido como solução para o problema do enunciado está abaixo:

	\lstinputlisting[language=C++, caption = Código fonte da prática 5 - armazena dados em arquivo .csv., captionpos=t]{Ex5.cpp}	
	
	O código fonte apresenta todos os itens solicitados. O nome do arquivo .csv para armazenamento dos dados é previamente definido pelo desenvolvedor (linha 40). Porém, o mesmo poderia ser solicitado ao usuário. Após a linha 45 é realizada a criação do arquivo com verificação de erro. No laço \emph{for} iniciado na linha 59, é iniciada a solicitação de entrada dos dados e consequente inscrição no arquivo .csv. Na linha 76, têm se a instrução de escrita no arquivo. 

\section[DISCUSSÃO DOS RESULTADOS]{Discussão dos resultados}

	Na figura~\ref{fig:ex5.1}, verifica-se a tela de execução do programa com os registros de entrada devidamente separados. 
	
	\begin{figure}[htp]
		\begin{center}
			\caption{Tela com a solicitação dos registros de entrada.}
			\includegraphics[width=0.8\textwidth]{Imgs/ex5.1.PNG}
			\label{fig:ex5.1}
		\end{center}
	\end{figure}

	Na figura ~\ref{fig:ex5.2}, verifica-se a criação do arquivo com o RU 3760288.csv e a correta inscrição dos dados de input separados por ponto e vírgula.

	\begin{figure}[htp]
		\begin{center}
			\caption{Tela com a criação do arquivo .csv e a inscrição dos dados de input.}
			\includegraphics[width=0.8\textwidth]{Imgs/ex5.2.PNG}
			\label{fig:ex5.2}
		\end{center}
	\end{figure}

% ----------------------------------------------------------
% Finaliza a parte no bookmark do PDF
% para que se inicie o bookmark na raiz
% e adiciona espaço de parte no Sumário
% ----------------------------------------------------------
%\phantompart

% ---
% Conclusão (outro exemplo de capítulo sem numeração e presente no sumário)
% ---
\chapter*[CONCLUSÃO]{Conclusão}
\addcontentsline{toc}{chapter}{CONCLUSÃO}
% ---
	Através da confecção desta atividade prática, o aluno teve a oportunidade de aprimorar a utilização da linguagem C e todos os conceitos apresentados na disciplina de Linguagem de programação.	Estruturas de repetição, estruturas de dados e funções recursivas são alguns dos pontos abordados nestes cinco exercícios práticos.

	Todos os exercícios foram desenvolvidos com a linguagem C no IDE de desenvolvimento Microsoft Visual Studio 2022(Community).

	Este documento e seu código-fonte foram produzidos em \LaTeX\ e são exemplos de referência de uso da classe \textsf{abntex2} e do pacote \textsf{abntex2cite}. O documento exemplifica a elaboração de trabalho acadêmico (tese, dissertação e outros do gênero) produzido conforme a ABNT NBR 14724:2011 \emph{Informação e documentação - Trabalhos acadêmicos - Apresentação}.


% \bibliography{abntex2-modelo-references}

\end{document}
