\documentclass[a4paper, 11pt]{article}
\usepackage{student}
\usepackage{amsmath,amssymb,bm}
\usepackage[brazilian]{babel}
\usepackage{enumerate}

% Metadata
\date{\today}
\setmodule{Movimento unidimensional - capítulo 2}
\setterm{Ano branco 1, 2022}

\title{Atividade}
\setmembername{Paulo Vinicius Pereira Pinheiro}  % Fill name
\setmemberuid{Movimento e dinâmica} 

\begin{document}
    \header{}
    \vspace{6pt}
    \textbf{Solucionar as questões indicadas abaixo extraídas do livro:}
    \begin{itemize}
        \item \textbf{Young \& Freedman, Sears \& Zemansky. Física 1 mecânica. 14 Ed. Pearson Ed. 2016.}
        \item Exercícios do capítulo 2:
        \begin{itemize}
            \item 2.1, 2.3, 2.9, 2.13, 2.21, 2.29, 2.31, 2.35, 2.37, 2.43, 2.47, 2.61, 2.65, 269.
        \end{itemize}
        \item Todas as questões selecionadas possuem suas respostas ao final do livro.
        \item \textcolor{red}{Atenção às respostas ao final do livro:} elas são compostas em sua maioria de valores arredondados. Considerar sempre valores aproximados.
        \item Fazer a entrega em ARQUIVO ÚNICO composto por imagens da resolução manuscrita ou arquivo digital no formato PDF. 
    \end{itemize}
    
    % \begin{question}
    %     Comente, do ponto de vista da lógica proposicional, a diferença entre sintaxe e semântica.
    % \end{question}
\end{document}