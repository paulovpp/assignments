% ------------------------------------------------------------------------
% == Modelo de Relatório Técnico/Acadêmico em conformidade com 
% == ABNT NBR 10719:2015 Informação e documentação -
% == Relatório técnico e/ou científico
% ------------------------------------------------------------------------
% == Adapatado para modelo do CPAI
% ------------------------------------------------------------------------

\documentclass[
	12pt,				  % tamanho da fonte
	%openright,	  % capítulos começam em pág ímpar (insere página vazia caso preciso)
	%twoside,			% para impressão em recto e verso. Oposto a oneside
  oneside,			% para impressão em recto e verso. Oposto a twoside
	a4paper,			% tamanho do papel. 
	english,			% idioma adicional para hifenização
	french,				% idioma adicional para hifenização
	spanish,			% idioma adicional para hifenização
	brazil,				% o último idioma é o principal do documento
	]{abntex2}

% ---
% Pacotes em pasta própria para deixar o documento mestre limpo
\usepackage{sty/pacotes}

% ---
% Informações de dados sobre o Relatório
% == IMPORTANTE ABRIR E ALTERAR OS DADOS DESTE ARQUIVO!
\usepackage{sty/info}

% Início do documento
\begin{document}
  % Seleciona o idioma do documento (conforme pacotes do babel)
  %\selectlanguage{english}
  \selectlanguage{brazil}
  
  % Retira espaço extra obsoleto entre as frases.
  \frenchspacing 
  
  % ----------------------------------------------------------
  % == ELEMENTOS PRÉ-TEXTUAIS
  % ----------------------------------------------------------
  % \pretextual
  
  % ---
  % Capa
  \imprimircapa
  
  % ---
  % Folha de rosto (o * indica que haverá a ficha bibliográfica)
  \imprimirfolhaderosto*
  
  % ---
  % Anverso da folha de rosto
%  %----------------------------------------------------------------------------------------------------------------
% File : creditos.tex
%----------------------------------------------------------------------------------------------------------------

% ---
% Anverso da folha de rosto:

{
  \ABNTEXchapterfont

  \vspace*{\fill}
  
  Conforme a ABNT NBR 10719:2015, seção 4.2.1.1.1, o anverso da folha de rosto deve conter:
  
  \begin{alineas}
    \item nome do órgão ou entidade responsável que solicitou ou gerou o relatório; 
    \item título do projeto, programa ou plano que o relatório está relacionado;
    \item título do relatório;
    \item subtítulo, se houver, deve ser precedido de dois pontos, evidenciando a sua subordinação ao título. O relatório em vários volumes deve ter um título geral. Além deste, cada volume pode ter um título específico; 
    \item número do volume, se houver mais de um, deve constar em cada folha de rosto a especificação do respectivo volume, em algarismo arábico; 
    \item código de identificação, se houver, recomenda-se que seja formado pela sigla da instituição, indicação da categoria do relatório, data, indicação do assunto e número sequencial do relatório na série; 
    \item classificação de segurança. Todos os órgãos, privados ou públicos, que desenvolvam pesquisa de interesse nacional de conteúdo sigiloso, devem informar a classificação adequada, conforme a legislação em vigor; 
    \item nome do autor ou autor-entidade. O título e a qualificação ou a função do autor podem ser incluídos, pois servem para indicar sua autoridade no assunto. Caso a instituição que solicitou o relatório seja a mesma que o gerou, suprime-se o nome da instituição no campo de autoria; 
    \item local (cidade) da instituição responsável e/ou solicitante; NOTA: No caso de cidades homônimas, recomenda-se o acréscimo da sigla da unidade da federação.
    \item ano de publicação, de acordo com o calendário universal (gregoriano), deve ser apresentado em algarismos arábicos.
  \end{alineas}

  \imprimirinstituicao
  
  \textbf{Projeto: \projeto}
  
  \imprimirtitulo
  
  
  
  \vspace*{\fill}
}

  %----------------------------------------------------------------------------------------------------------------
% File : ficha.tex
%----------------------------------------------------------------------------------------------------------------

% ---
% Anverso da folha de rosto:

{
  \ABNTEXchapterfont\setlength{\parindent}{0cm}
  
  \vspace*{\fill} 
  
  \copyright\ CPAI 2016
  
  Classificação de Segurança: 10  -- Restrito
  
  \vspace*{\fill} % Posição vertical
  
  \begin{fichacatalografica}\ABNTEXchapterfont
    \vspace*{\fill} % Posição vertical
    \begin{center}
      {\scriptsize Dados Internacionais de Catalogação na Publicação (CIP)} 
      
      {\scriptsize (Câmara Brasileira do Livro, SP, Brasil)} 
    \end{center}
    \hrule % Linha horizontal
    \begin{center} % Minipage Centralizado
      \begin{minipage}[c]{13cm} % Largura
        \imprimirautor
        
        \hspace{0.5cm} \projeto\ : \imprimirtitulo\ / \imprimirautor. --
        \imprimirlocal : \UNB : \CPAI, \imprimirdata.
        
        \hspace{0.5cm} \pageref{LastPage} p. : il. (algumas color.) ; 29,7 cm.\\
        
        \hspace{0.5cm}
        \parbox[t]{\textwidth}{\imprimirtipotrabalho~--~\CPAI, \imprimirdata.}
        
        \hspace{0.5cm}
        \parbox[t]{\textwidth}{Versão \versao\ (final).}\\
        
        %\hspace{0.5cm}
        %\parbox[t]{\textwidth}{ISSN: \imprimirissn}\\
        
        \hspace{0.5cm}
        \kwords\ I. Título
        \begin{flushright}
          CDD 22
        \end{flushright}
      \end{minipage}
    \end{center}
    \hrule
  \end{fichacatalografica}
  
  \cleardoublepage
}

%----------------------------------------------------------------------
  
  % ---
  % Agradecimentos
  %----------------------------------------------------------------------------------------------------------------
% File : agradecimentos.tex
%----------------------------------------------------------------------------------------------------------------

\begin{agradecimentos}
  O agradecimento principal é direcionado a Youssef Cherem, autor do \nameref{formulado-identificacao} (\autopageref{formulado-identificacao}).
  
  Os agradecimentos especiais são direcionados ao Centro de Pesquisa em Arquitetura da Informação\footnote{\url{http://www.cpai.unb.br/}} da Universidade de Brasília (CPAI), ao grupo de usuários \emph{latex-br}\footnote{\url{http://groups.google.com/group/latex-br}} e aos novos voluntários do grupo \emph{\abnTeX}\footnote{\url{http://groups.google.com/group/abntex2} e \url{http://www.abntex.net.br/}}~que contribuíram e que ainda contribuirão para a evolução do abn\TeX.
  
\end{agradecimentos}

  
  % ---
  % Resumo
  \include{pretextuais/resumo}
  
  % ---
  % inserir listas de ilustrações, tabelas e quadros
  \include{pretextuais/itq}
  
  % ---
  % inserir listas de siglas
  \include{pretextuais/siglas}
  
  % ---
  % inserir o sumario
  \pdfbookmark[0]{\contentsname}{toc}
  \tableofcontents*
  \cleardoublepage
  % ---

  % ----------------------------------------------------------
  % == ELEMENTOS TEXTUAIS
  \textual
  
  % ----------------------------------------------------------
  % Introdução
  \include{textuais/introducao}
  
  % ----------------------------------------------------------
  % == PARTE - preparação da pesquisa
  % ----------------------------------------------------------
  \part{Preparação do relatório}
  
  % ----------------------------------------------------------
  % Capitulo com exemplos de comandos inseridos de arquivo externo 
  \include{textuais/exemplo}
  
  % ----------------------------------------------------------
  % == PARTE - Resultados
  % ----------------------------------------------------------
  \part{Resultados}
  
  % ---
  % Capitulo de revisão de literatura
  %----------------------------------------------------------------------------------------------------------------
% File : rev_literatura.tex
%----------------------------------------------------------------------------------------------------------------

% ---
% Capitulo de revisão de literatura
\chapter{Lorem ipsum dolor sit amet}

% ---
\section{Aliquam vestibulum fringilla lorem}
\lipsum[1]        % Gera texto aleatório

\lipsum[2-3]      % Gera texto aleatório

  
  % ---
  % == PARTE - Finaliza a parte no bookmark do PDF
  %    para que se inicie o bookmark na raiz e adiciona espaço de parte no Sumário
  \phantompart
  
  % ---
  % Conclusão
  %----------------------------------------------------------------------------------------------------------------
% File : conclusao.tex
%----------------------------------------------------------------------------------------------------------------

% ---
% Conclusão
\chapter{Conclusão}
\lipsum[31-33]

  
  % ----------------------------------------------------------
  % == ELEMENTOS PÓS-TEXTUAIS
  % ----------------------------------------------------------
  \postextual
  
  % ----------------------------------------------------------
  % Referências bibliográficas
  \bibliography{bib/CPAI,bib/abntex2-modelo-references}
  
  % ----------------------------------------------------------
  % Glossário
  % Consulte o manual da classe abntex2 para orientações sobre o glossário.
  %\glossary
  
  % ----------------------------------------------------------
  % == APÊNDICES
  % ----------------------------------------------------------
  
  % ---
  % Inicia os apêndices
  \begin{apendicesenv}
  
    % Imprime uma página indicando o início dos apêndices
    \partapendices
    
    % Apendices (incluir os arquivos com os apêndices)
    \include{postextuais/apendice1}
  
  \end{apendicesenv}
  
  % ----------------------------------------------------------
  % == ANEXOS
  % ----------------------------------------------------------
  
  % ---
  % Inicia os anexos
  % ---
  \begin{anexosenv}
    
    % Imprime uma página indicando o início dos anexos
    \partanexos
    
    % ---
    % Anexos (incluir os arquivos com os anexos)
    \include{postextuais/anexo1}
  
  \end{anexosenv}
  
  %---------------------------------------------------------------------
  % == INDICE REMISSIVO
  %---------------------------------------------------------------------
  \phantompart
  
  \printindex
  
  %---------------------------------------------------------------------
  % Formulário de Identificação (opcional)
  %---------------------------------------------------------------------
  %%----------------------------------------------------------------------------------------------------------------
% File : form_identif.tex
%----------------------------------------------------------------------------------------------------------------

%---------------------------------------------------------------------
% Formulário de Identificação (opcional)
%---------------------------------------------------------------------
\chapter*[Formulário de Identificação]{Formulário de Identificação}
\addcontentsline{toc}{chapter}{Exemplo de Formulário de Identificação}
\label{formulado-identificacao}

Exemplo de Formulário de Identificação, compatível com o Anexo A (informativo) da ABNT NBR 10719:2015. Este formulário não é um anexo. Conforme definido na norma, ele é o último elemento pós-textual e opcional do relatório.

\bigskip

\begin{longtable}{|p{9cm}|p{5cm}|}
  \hline
  \multicolumn{2}{|c|}{\textbf{\large Dados do Relatório de Pesquisa}}\\
  \hline \endfirsthead

  \hline
  \multicolumn{2}{|c|}{\textbf{\large Dados do Relatório de Pesquisa}}\\
  \hline \endhead

  \multirow{4}{10cm}[24pt]{Título e subtítulo}& Classificação de segurança\\
  & \\
  \cline{2-2}
  & No.\\
  & \\
  
  \hline
  Tipo de relatório & Data\\
  \hline
  Título do projeto/programa/plano & No.\\
  \hline
  \multicolumn{2}{|l|}{Autor(es)} \\
  \hline
  \multicolumn{2}{|l|}{Instituição executora e endereço completo} \\
  \hline
  \multicolumn{2}{|l|}{Instituição patrocinadora e endereço completo} \\
  \hline
  \multicolumn{2}{|l|}{Resumo}\\[3cm]
  \hline
  \multicolumn{2}{|l|}{Palavras-chave/descritores}\\
  \hline
  \multicolumn{2}{|l|}{
    Edição \hfill No. de páginas \hfill No. do volume \hfill Nº de classificação \phantom{XXXX}} \\
  \hline
  \multicolumn{2}{|l|}{
    ISSN \hfill \hfill Tiragem \hfill Preço \phantom{XXXXXXXX}} \\
  \hline
  \multicolumn{2}{|l|}{Distribuidor} \\
  \hline
  \multicolumn{2}{|l|}{Observações/notas}\\[3cm]
  \hline
\end{longtable}


\end{document}
