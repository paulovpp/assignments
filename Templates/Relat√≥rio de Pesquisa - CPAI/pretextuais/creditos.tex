%----------------------------------------------------------------------------------------------------------------
% File : creditos.tex
%----------------------------------------------------------------------------------------------------------------

% ---
% Anverso da folha de rosto:

{
  \ABNTEXchapterfont

  \vspace*{\fill}
  
  Conforme a ABNT NBR 10719:2015, seção 4.2.1.1.1, o anverso da folha de rosto deve conter:
  
  \begin{alineas}
    \item nome do órgão ou entidade responsável que solicitou ou gerou o relatório; 
    \item título do projeto, programa ou plano que o relatório está relacionado;
    \item título do relatório;
    \item subtítulo, se houver, deve ser precedido de dois pontos, evidenciando a sua subordinação ao título. O relatório em vários volumes deve ter um título geral. Além deste, cada volume pode ter um título específico; 
    \item número do volume, se houver mais de um, deve constar em cada folha de rosto a especificação do respectivo volume, em algarismo arábico; 
    \item código de identificação, se houver, recomenda-se que seja formado pela sigla da instituição, indicação da categoria do relatório, data, indicação do assunto e número sequencial do relatório na série; 
    \item classificação de segurança. Todos os órgãos, privados ou públicos, que desenvolvam pesquisa de interesse nacional de conteúdo sigiloso, devem informar a classificação adequada, conforme a legislação em vigor; 
    \item nome do autor ou autor-entidade. O título e a qualificação ou a função do autor podem ser incluídos, pois servem para indicar sua autoridade no assunto. Caso a instituição que solicitou o relatório seja a mesma que o gerou, suprime-se o nome da instituição no campo de autoria; 
    \item local (cidade) da instituição responsável e/ou solicitante; NOTA: No caso de cidades homônimas, recomenda-se o acréscimo da sigla da unidade da federação.
    \item ano de publicação, de acordo com o calendário universal (gregoriano), deve ser apresentado em algarismos arábicos.
  \end{alineas}

  \imprimirinstituicao
  
  \textbf{Projeto: \projeto}
  
  \imprimirtitulo
  
  
  
  \vspace*{\fill}
}
