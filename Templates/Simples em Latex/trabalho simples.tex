\documentclass[
	% -- opções da classe memoir --
	12pt,				% tamanho da fonte
	openright,			% capítulos começam em pág ímpar (insere página vazia caso preciso)
	oneside,			% para impressão em verso e anverso. Oposto a oneside
	a4paper,			% tamanho do papel. 
	% -- opções da classe abntex2 --
	%chapter=TITLE,		% títulos de capítulos convertidos em letras maiúsculas
	%section=TITLE,		% títulos de seções convertidos em letras maiúsculas
	%subsection=TITLE,	% títulos de subseções convertidos em letras maiúsculas
	%subsubsection=TITLE,% títulos de subsubseções convertidos em letras maiúsculas
	% -- opções do pacote babel --
	english,			% idioma adicional para hifenização
	french,				% idioma adicional para hifenização
	spanish,			% idioma adicional para hifenização
	brazil				% o último idioma é o principal do documento
	]{abntex2}

% ---
% Pacotes básicos 
% ---
\usepackage{lmodern}			% Usa a fonte Latin Modern			
\usepackage[T1]{fontenc}		% Selecao de codigos de fonte.
\usepackage[utf8]{inputenc}		% Codificacao do documento (conversão automática dos acentos)
\usepackage{lastpage}			% Usado pela Ficha catalográfica
\usepackage{indentfirst}		% Indenta o primeiro parágrafo de cada seção.
\usepackage{color}				% Controle das cores
\usepackage{graphicx}			% Inclusão de gráficos
\usepackage{microtype} 			% para melhorias de justificação
\usepackage[brazilian,hyperpageref]{backref}	 % Paginas com as citações na bibl
\usepackage[alf]{abntex2cite}	% Citações padrão Absent
\usepackage{lipsum}

% alterando o aspecto da cor azul
\definecolor{blue}{RGB}{41,5,195}

% informações do PDF
\makeatletter
\hypersetup{
    %pagebackref=true,
	pdftitle={\@title}, 
	pdfauthor={{Name}},
	pdfsubject={\imprimirpreambulo},
	pdfcreator={LaTeX with abnTeX2},
	pdfkeywords={abnt}{latex}{abntex}{abntex2}{trabalho acadêmico}, 
	colorlinks=true,       		% false: boxed links; true: colored links
	linkcolor=blue,          	% color of internal links
	citecolor=blue,        		% color of links to bibliography
	filecolor=magenta,      		% color of file links
	urlcolor=blue,
	bookmarksdepth=4
}
\makeatother

% O tamanho do parágrafo é dado por:
\setlength{\parindent}{1.3cm}

% Controle do espaçamento entre um parágrafo e outro:
\setlength{\parskip}{0.2cm}  % tente também \onelineskip

% compila o indice
% ---
\makeindex
% ----
% Início do documento
% ----
\begin{document}

% Retira espaço extra obsoleto entre as frases.
\frenchspacing 

% Informações de dados para CAPA
% Capa sem o uso de \capa
\begin{capa}
	\begin{center}
		\ABNTEXchapterfont\large{
			\includegraphics[scale=1]{logo.png}\\
			CENTRO UNIVERSITÁRIO INTERNACIONAL UNINTER\\
			ESCOLA SUPERIOR POLITÉCNICA\\
			BACHARELADO EM ENGENHARIA DA COMPUTAÇÃO\\
			DISCIPLINA DE LINGUAGEM DE PROGRAMAÇÃO
		}
		\vfill

		\ABNTEXchapterfont\bfseries\LARGE{
			Modelo Canônico de Trabalho Acadêmico com \abnTeX2
		}
	
		\vspace*{2cm}

		\begin{flushright}
			\ABNTEXchapterfont\bfseries\normalsize{
				Autor: Paulo Vinicius Pereira Pinheiro\\
				No. RU: 3760288\\
				Prof. Lauro César Araujo
			}
		\end{flushright}
		
		\vfill

		% \ABNTEXchapterfont\large{
		\ABNTEXchapterfont\normalsize{
			Juazeiro do Norte - Ceará\\
			2022			
		}
	\end{center}
\end{capa}

% resumo em português
% \setlength{\absparsep}{18pt} % ajusta o espaçamento dos parágrafos do resumo
% \begin{resumo}
%  Segundo a \citeonline[3.1-3.2]{NBR6028:2003}, o resumo deve ressaltar o
%  objetivo, o método, os resultados e as conclusões do documento. A ordem e a extensão
%  destes itens dependem do tipo de resumo (informativo ou indicativo) e do
%  tratamento que cada item recebe no documento original. O resumo deve ser
%  precedido da referência do documento, com exceção do resumo inserido no
%  próprio documento. (\ldots) As palavras-chave devem figurar logo abaixo do
%  resumo, antecedidas da expressão Palavras-chave:, separadas entre si por
%  ponto e finalizadas também por ponto.

%  \textbf{Palavras-chaves}: latex. abntex. editoração de texto.
% \end{resumo}

% resumo em inglês
% \begin{resumo}[Abstract]
%  \begin{otherlanguage*}{english}
%    This is the english abstract.

%    \vspace{\onelineskip}
 
%    \noindent 
%    \textbf{Key-words}: latex. abntex. text editoration.
%  \end{otherlanguage*}
% \end{resumo}

% resumo em francês 
% \begin{resumo}[Résumé]
%  \begin{otherlanguage*}{french}
%     Il s'agit d'un résumé en français.
 
%    \textbf{Mots-clés}: latex. abntex. publication de textes.
%  \end{otherlanguage*}
% \end{resumo}

% % resumo em espanhol
% \begin{resumo}[Resumen]
%  \begin{otherlanguage*}{spanish}
%    Este es el resumen en español.
  
%    \textbf{Palabras clave}: latex. abntex. publicación de textos.
%  \end{otherlanguage*}
% \end{resumo}
% ---

% ---
% inserir lista de ilustrações
% ---
% \pdfbookmark[0]{\listfigurename}{lof}
% \listoffigures*
% \cleardoublepage
% % ---

% % ---
% % inserir lista de tabelas
% % ---
% \pdfbookmark[0]{\listtablename}{lot}
% \listoftables*
% \cleardoublepage
% % ---

% % ---
% % inserir lista de abreviaturas e siglas
% % ---
% \begin{siglas}
%   \item[ABNT] Associação Brasileira de Normas Técnicas
%   \item[abnTeX] ABsurdas Normas para TeX
% \end{siglas}
% % ---

% % ---
% % inserir lista de símbolos
% % ---
% \begin{simbolos}
%   \item[$ \Gamma $] Letra grega Gama
%   \item[$ \Lambda $] Lambda
%   \item[$ \zeta $] Letra grega minúscula zeta
%   \item[$ \in $] Pertence
% \end{simbolos}
% % ---

% ---
% inserir o sumario
% ---
\pdfbookmark[0]{\contentsname}{toc}
\tableofcontents*
% \cleardoublepage
% ---



% ----------------------------------------------------------
% ELEMENTOS TEXTUAIS
% ----------------------------------------------------------
\textual

% ----------------------------------------------------------
% Introdução (exemplo de capítulo sem numeração, mas presente no Sumário)
% ----------------------------------------------------------
\chapter*[Introdução]{Introdução}
\addcontentsline{toc}{chapter}{Introdução}
% ----------------------------------------------------------

Este documento e seu código-fonte são exemplos de referência de uso da classe
\textsf{abntex2} e do pacote \textsf{abntex2cite}. O documento 
exemplifica a elaboração de trabalho acadêmico (tese, dissertação e outros do
gênero) produzido conforme a ABNT NBR 14724:2011 \emph{Informação e documentação
- Trabalhos acadêmicos - Apresentação}.

A expressão ``Modelo Canônico'' é utilizada para indicar que \abnTeX\ não é
modelo específico de nenhuma universidade ou instituição, mas que implementa tão
somente os requisitos das normas da ABNT. Uma lista completa das normas

Sinta-se convidado a participar do projeto \abnTeX! Acesse o site do projeto em
\url{http://abntex2.googlecode.com/}. Também fique livre para conhecer,
estudar, alterar e redistribuir o trabalho do \abnTeX, desde que os arquivos
modificados tenham seus nomes alterados e que os créditos sejam dados aos
autores originais, nos termos da ``The \LaTeX\ Project Public
License''\footnote{\url{http://www.latex-project.org/lppl.txt}}.

Encorajamos que sejam realizadas customizações específicas deste exemplo para
universidades e outras instituições --- como capas, folha de aprovação, etc.
Porém, recomendamos que ao invés de se alterar diretamente os arquivos do
\abnTeX, distribua-se arquivos com as respectivas customizações.
Isso permite que futuras versões do \abnTeX~não se tornem automaticamente
incompatíveis com as customizações promovidas. Consulte


Esperamos, sinceramente, que o \abnTeX\ aprimore a qualidade do trabalho que
você produzirá, de modo que o principal esforço seja concentrado no principal:
na contribuição científica.

Equipe \abnTeX 

Lauro César Araujo


% ----------------------------------------------------------
% PARTE
% ----------------------------------------------------------
% \part{Preparação da pesquisa}
% ----------------------------------------------------------

% ---
% Capitulo com exemplos de comandos inseridos de arquivo externo 
% ---
% \include{abntex2-modelo-include-comandos}
% ---

% ----------------------------------------------------------
% PARTE
% ----------------------------------------------------------
% \part{Referenciais teóricos}
% ----------------------------------------------------------

% ---
% Capitulo de revisão de literatura
% ---
\chapter{Lorem ipsum dolor sit amet}
% ---

% ---
\section{Aliquam vestibulum fringilla lorem}
% ---

\lipsum[1]

\lipsum[2-3]

% ----------------------------------------------------------
% PARTE
% ----------------------------------------------------------
% \part{Resultados}
% ----------------------------------------------------------

% ---
% primeiro capitulo de Resultados
% ---
\chapter{Lectus lobortis condimentum}
% ---

% ---
\section{Vestibulum ante ipsum primis in faucibus orci luctus et ultrices
posuere cubilia Curae}
% ---

\lipsum[21-22]

% ---
% segundo capitulo de Resultados
% ---
\chapter{Nam sed tellus sit amet lectus urna ullamcorper tristique interdum
elementum}
% ---

% ---
\section{Pellentesque sit amet pede ac sem eleifend consectetuer}
% ---

\lipsum[24]

% ----------------------------------------------------------
% Finaliza a parte no bookmark do PDF
% para que se inicie o bookmark na raiz
% e adiciona espaço de parte no Sumário
% ----------------------------------------------------------
%\phantompart

% ---
% Conclusão (outro exemplo de capítulo sem numeração e presente no sumário)
% ---
\chapter*[Conclusão]{Conclusão}
\addcontentsline{toc}{chapter}{Conclusão}
% ---

\lipsum[31-33]

% \bibliography{abntex2-modelo-references}

\end{document}
