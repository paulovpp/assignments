\documentclass[a4paper,pra,aps,twocolumn,superscriptaddress,10pt]{revtex4-2}
\usepackage[pretty,uselistings]{revquantum}
% \usepackage[brazil]{babel}
\usepackage[T1]{fontenc}
\usepackage[utf8]{inputenc}
\usepackage{stmaryrd} 
\SetSymbolFont{stmry}{bold}{U}{stmry}{m}{n} 
\usepackage{bm}
\usepackage{amsmath}
\usepackage{amssymb}
\usepackage{amsfonts}
\usepackage{silence}
\WarningFilter{revtex4-2}{Repair the float}
\usepackage{anyfontsize}
\usepackage{lipsum}

%=============================================================================
% FRONT MATTER
%=============================================================================

\begin{document}

\title{Atividade prática da disciplina de Matemática Computacional}

\author{Paulo Vinicius Pereira Pinheiro}
\email{paulovpp@gmail.com}
\affiliation{UNINTER - Centro Universitário Internacional}
\affiliation{RU: 3760288}

\date{\today}

\begin{abstract}
    A crescente necessidade por segurança nas transações online traz à tona um problema de grande complexidade: até quando os protocolos atuais de segurança da informação são capazes de nos manter seguros? Inúmeros são os métodos utilizados para comunicação segura. Dentre eles, destaca-se a criptografia. De forma simétrica ou assimétrica, o ato de criptografar uma mensagem é torná-la inelegível a aqueles à quem a mensagem não se destina. A criptografia é um dos tópicos abordados na disciplina de matemática computacional do curso de engenharia da computação da UNINTER. E o trabalho abaixo realizado trata do assunto trabalha a criptografia de forma aplicada. São solicitados os seguintes procedimentos: que sejam codificadas as 4 primeiras letras do nome do aluno utilizando-se a cifra simétrica de Feistel com apenas 2 estágios, utilizando o último dígito do RU, $K$ como a chave criptográfica para ambos os estágios. Deve-se utilizar também como função $F$ um shift left, cíclico, de $K$ posições. Solicita-se também que a mensagem codificada seja preparada para transmissão e que, em seguida, haja o processo de decodificação, comprovando assim a reciprocidade do processo.
    
    
\end{abstract}

\maketitle

%=============================================================================
% MAIN DOCUMENT
%=============================================================================

\section{Introdução}

    \lipsum[2-4]

    é chamado de cifragem. A cifragem é um processo de transformação de um texto em uma sequência de caracteres que é conhecida como uma cifra. A cifra é um método de criptografia que consiste em transformar uma sequência de caracteres em outra sequência de caracteres, ou seja, é um processo de substituição.
\section{Desenvolvimento}
    \lipsum[2-4]


\section{Conclusão}
    \lipsum[2-4]

% \bibliography{example}

%=============================================================================
% APPENDICES
%=============================================================================

\appendix

%=============================================================================
\section{Source Code}
\label{apx:code}
%=============================================================================

\begin{lstlisting}[style=python,gobble=4,caption={Quick example of importing}]
        import qutip as qt
\end{lstlisting}

\end{document} 